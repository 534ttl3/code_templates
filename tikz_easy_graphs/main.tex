\documentclass{article}
\usepackage[top=2cm, left=2cm, right=2cm, bottom=2cm]{geometry}

\usepackage{tikz}
\usepackage{varwidth}
\usetikzlibrary{graphs,graphdrawing,
graphdrawing.force,graphdrawing.layered}
%\usegdlibrary{force}

\def\annot#1{node[above, sloped] {#1}}
\begin{document}

\section*{Tikz graph drawing with automatic layout}
\subsection*{Force-directed graph drawing, \textit{spring layout}}
\tikz [spring layout, node distance=1in] {
\node (a) {Hello};
\node (b) {World};
\node (c) {$ c^2 $};
\node (d) {$ \delta $};
\node (e) [draw, text width=3cm] {Using fixed width minipage};
\node (f) [draw, align=left] {\begin{varwidth}{5cm} using varwidth package \end{varwidth}};

\graph {
    (a) -- (c);
    (b) -- (c);
    (b) -- (d);
    (b) -> (e);
    (e) -> (f);
}
}

% ---------------

\subsection*{Layered graph drawing, \textit{layered layout}}

\tikz [layered layout, layer distance=3cm, 
every edge/.append style={every node/.style={sloped, above}}] {
\node (a) {Hello};
\node (b) {World};
\node (c) {$ c^2 $};
\node (d) [yshift=2cm] {$ \delta $};
\node (e) [draw, fill=white, opacity=0.7] {\begin{varwidth}{5cm} \narrowragged This is a demonstration of how line breaks work with more words $$ \frac{\mathrm{d}\langle O \rangle}{\mathrm{d}t} = \langle [H, O]\rangle + \partial_t O $$ and an equation. \end{varwidth}};
\node (f) {Fellow};

\draw 
(a) edge[->] node {Hey} (b) 
(a) edge[->] node {my dear} (f)
(b) edge[->, out=0, in=60] (f)
(b) edge[->] (c)
(b) edge[->] (d)
(b) edge[--] (e)
(e) edge[--] (f);
}

\end{document}
